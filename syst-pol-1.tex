\begin{frame}{Systematic uncertainties due to $\chib$ polarization}
\begin{block}{}
\small
Efficiencies are evaluated on MC where \chib particles are unpolarized. To evaluate systematic effects due to the unknown polarization of \chib, MC events are reweighted as described in \textcolor{blue}{HERA-B Collaboration, I. Abt et al.,  Production of the Charmonium States $\chi_{c1}$ and $\chi_{c2}$ in Proton Nucleus Interactions at s = 41.6-GeV}, \href{http://arxiv.org/abs/0807.2167}{arXiv:0807.2167} and
\textcolor{blue}{LHCb collaboration, R. Aaij et al.,Measurement of the cross-section ratio $\sigma(\chi_{c2})/\sigma(\chi_{c1})$ for prompt $\chi_c$
production at $\sqrt{s}=7$ TeV}, \href{http://arxiv.org/abs/1202.1080}{arXiv:1202.1080}
\end{block}
For each simulated event in the unpolarised sample, a weight is calculated from
the distribution of the following angles in the various polarisation hypotheses compared to the
unpolarised distribution.

\scalebox{0.7}{
\begin{tabular}{c|p{10cm}}\toprule
$\Theta_{\Upsilon}$ & angle between the directions of the $\mu^{+}$ in the $\Upsilon$ rest frame and the $\Upsilon$ in the $\chi_b$ rest frame.\\
\midrule
$\Theta_{\chi_b}$ & angle between the directions of the $\Upsilon$ in the $\chi_b$ rest frame and $\chi_b$ in the lab frame.\\
\midrule
$\phi$ & angle between the $\Upsilon$ decay plane in the $\chi_b$ rest frame and the plane formed by $\chi_b$ direction in the lab frame and the direction of the $\Upsilon$ in the $\chi_b$ rest frame.\\
\bottomrule
\end{tabular}
}

\vspace{0.1in}

Two hypotheses for $\chibone$ state (w0, w1) and three hypotheses for $\chibtwo$ (w0, w1, w2).

\end{frame}